\documentclass[runningheads,a4paper]{llncs}
\usepackage{color}
\usepackage{amssymb}
\setcounter{tocdepth}{3}
\usepackage{graphicx}
\usepackage{framed}
\usepackage{url}
\usepackage{float}
\urldef{\mailsa}\path|firstname.lastname@gesis.org|
\newcommand{\keywords}[1]{\par\addvspace\baselineskip
\noindent\keywordname\enspace\ignorespaces#1}

\begin{document}

\mainmatter  % start of an individual contribution

% first the title is needed
\title{Analyzing the research output presented at European Networked Knowledge Organization Systems workshops (2000-2015)}

% a short form should be given in case it is too long for the running head


% the name(s) of the author(s) follow(s) next
%
% NB: Chinese authors should write their first names(s) in front of
% their surnames. This ensures that the names appear correctly in
% the running heads and the author index.
%
\author{Fakhri Momeni%
	\and Philipp Mayr}
%
\titlerunning{Analyzing the NKOS research output (2000-2015)}
% (feature abused for this document to repeat the title also on left hand pages)

% the affiliations are given next; don't give your e-mail address
% unless you accept that it will be published
\author{Fakhri Momeni and Philipp Mayr}
\institute{GESIS - Leibniz Institute for the Social Sciences,\\
	Unter Sachsenhausen 6-8\\
	50667 Cologne, Germany\\
	\email{firstname.lastname@gesis.org} }

%
% NB: a more complex sample for affiliations and the mapping to the
% corresponding authors can be found in the file "llncs.dem"
% (search for the string "\mainmatter" where a contribution starts).
% "llncs.dem" accompanies the document class "llncs.cls".
%

%\toctitle{Lecture Notes in Computer Science}
%\tocauthor{Authors' Instructions}
\maketitle


\begin{abstract}		
In this paper we analyse a major part of the research output of the Networked Knowledge Organization Systems (NKOS) community in the period 2000 to 2015. We focus on the paper output presented at the European NKOS workshops in the last 15 years. Our open dataset "the NKOS bibliography" includes 14 workshop programmes (ECDL 2000-2010, TPDL 2011-2015) and 4 special  issues on NKOS (2001, 2004, 2006 and 2015) which cover in total 171 %add number
papers with 218 %add number
single authors. A focus of the analysis is the development of collaboration, core authors and topics in this interdisciplinary field. %be more precise

%add some more facts
 
\keywords{NKOS workshops, Special issues, Output analysis, Network analysis, Central authors, Collaboration}
\end{abstract}


\section{Introduction}\label{intro}
%intro and motivation %philipp

The European NKOS network has held a long-running series of annual workshops at the European Conference on Digital Libraries (ECDL), latterly reformed as the International Conference on Theory and Practice of Digital Libraries (TPDL). 
Typically, recent advances of KOS have been reported at the NKOS workshops, e.g. including the Simple Knowledge Organization System (SKOS) W3C standard, the ISO 25964 thesauri standard, the CIDOC Conceptual Reference Model (CRM), Linked Data applications, KOS-based recommender systems, KOS mapping techniques, KOS registries and metadata, social tagging, user-centred issues, and many other topics. A comprehensive and well cited review article on KOS was published in 2004 \cite{Zeng2004}. Special issues on Networked Knowledge Organization Systems (NKOS) have been published in Journal of Digital Information in 2001 and 2004, in New Review of Hypermedia and Multimedia in 2006 and in the International Journal of Digital Libraries in 2015 \cite{Mayr2016}. 

The motivation of this paper is to analyse the research output of the NKOS community. We are focusing on the informal part of this output, the presentations given the workshops. The specialty of this output is that these research papers typically are not published in journals or conference proceedings, these papers appear just as oral presentations at the workshop and are documented on the website. To our knowledge nobody has done an analysis on this part of the research output before. 



\section{NKOS workshop bibliography}\label{dataset}
%data set description %philipp

For our analysis we have compiled an open dataset the "NKOS bibliography"\footnote{The NKOS workshop bibliography is maintained in the following github repository: https://github.com/PhilippMayr/NKOS-bibliography} which includes 14 workshop programmes with all presented papers at ECDL 2000, 2003-2010 and TPDL 2011-2015. We added papers from 4 special issues on NKOS which have been published in the same period.

In a first step we have extracted all paper titles presented at the NKOS workshop websites. We manually disambiguated author names. We added the papers from the special issues. These paper are the only formal publications in our analysis.

Our dataset covers in total n %add number
papers with n %add number
single authors.

Table provides an overview of all workshop papers.

%include an overview table (venue;# of papers;# of authors; av # of authors)

\begin{tabular}	{|c|c|c|}
	\hline 
	venue& papers  & authors  \\ 
	\hline 
	ECDL 2000&  &  \\ 
	\hline 
	ECDL 2003&  &  \\ 
	\hline 
	ECDL 2004&  &  \\ 
	\hline 
	ECDL 2005&  &  \\ 
	\hline 
	ECDL 2006&  &  \\ 
	\hline 
	ECDL 2007&  &  \\ 
	\hline 
	ECDL 2008&  &  \\ 
	\hline 
	ECDL 2009&  &  \\ 
	\hline 
	ECDL 2010&  &  \\ 
	\hline 
	TPDL 2011&  &  \\ 
	\hline 
	TPDL 2012&  &  \\ 
	\hline 
	TPDL 2013&  &  \\ 
	\hline 
	TPDL 2014&  &  \\ 
	\hline 
	TPDL 2015&  &  \\ 
	\hline 
\end{tabular} 


Table provides an overview of all papers in the special issues.

\begin{tabular}{|c|c|c|}
	\hline 
	venue& papers  & authors  \\ 
	\hline 
	SI 2001&  &  \\ 
	\hline 
	ECDL 2004&  &  \\ 
	\hline 
	ECDL 2006&  &  \\ 
	\hline 
	ECDL 2015&  &  \\ 
	\hline 
\end{tabular} 


%The most cited NKOS papers in this time: Zeng, Dörr ... Koch 



\section{Analysis}\label{analysis}
In order to analyze the collaboration in the community of NKOS workshops we built a network of all authors in this community and computed the centrality of each author. For this purpose we utilized some centrality measures in Pajek\footnote{A program for analysis and visualization of very large networks(http://mrvar.fdv.uni-lj.si/pajek/)} . The network is composed  of pairs of author names. Each pair means that two authors cooperated for writing a paper. For a paper with n authors, there are n*(n-1) pairs. These pairs built the network for analysis in Pajek. to avoid repetition of pairs, we gave weight to pairs and it is equal to number of cooperations for two authors in different papers. 
Two used centrality measures of authors are degree and betweenness. Degree is the number of nodes that a focal node is connected to, and measures the involvement of the node in the network\cite{Opsahl2010}. In our authorship-network it specifies the sum of co-authors for all papers that each author has written. betweenness, assess the degree to which a node lies on the shortest path between two other nodes, and are able to funnel the flow in the network\cite{Opsahl2010}. In the authorship-network the author with high betweenness has a large influence on the transfer of information. 

%fakhri

\section{Results}\label{results}
Figure \ref{fig:wholenet} demonstrates a general view of the network. In this view each author has at least one co-author. From the network illustrated in this Figure we selected the largest component thats is represent in Figure\ref{fig:largestComponent}. 68 authors(31\% of all authors) are in this component.
\begin{figure}[H]
	\centering
	\includegraphics[width=1.0\linewidth]{wholeNet}
	\caption{The whole network of authors in community of NKOS workshops visualized in Pajek}
	\label{fig:wholenet}
\end{figure}
\begin{figure}[H]
	\centering
	\includegraphics[width=1\linewidth]{largestComponent}
	\caption{argest component in the NKOS authorship-network}
	\label{fig:largestComponent}
\end{figure}
To show the quantity of collaboration in the community we measured the degree centrality for each author. Figure\ref{fig:degreePercentage} shows Percentage of authors with different degrees. From the Figure we see that 15\% (with degree=0) of authors had no collaboration with others and 53\% of them had just maximum 6 cooperations with other authors. Only 32\% had with at least 8 co-authors for all their papers. \\
\begin{figure}[H]
	\centering
	\includegraphics[width=0.7\linewidth]{degreePercentage}
	\caption{}
	\label{fig:degreePercentage}
\end{figure}

Figure \ref{fig:degree16} shows the authors with most degree (more than 16) in the network. \\
\begin{figure}[H]
	\centering
	\includegraphics[width=0.7\linewidth]{degree16}
	\caption{Authors with degree more than 16 in the community of NKOS workshops}
	\label{fig:degree16}
\end{figure}
To detect the influence of authors on information exchange we calculated the betweenness centrality of authors. Figure \ref{fig:betweenness} indicates the authors with most betweenness (more than 0.001). comparing betweenness with degree we can see, the ranking of authors has changed in betweenness. Some authors despite their high cooperation with other authors, have lower ranking in betweenness comporison to others with lower degree. 

\begin{figure}
\centering
\includegraphics[width=0.7\linewidth]{betweenness}
\caption{Authors with betweenness more than 0.001 in the community of NKOS workshops}
\label{fig:betweenness}
\end{figure}






\section{Conclusion}\label{concl}
...%both

\section{Acknowledgment}\label{sec:ACKNOWLEDGMENTS}
...

\newpage

\bibliographystyle{splncs03} % abbrev
\bibliography{nkos} 

\end{document}
