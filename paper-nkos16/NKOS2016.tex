\documentclass[runningheads,a4paper]{llncs}
\usepackage{color}
\usepackage{amssymb}
\setcounter{tocdepth}{3}
\usepackage{graphicx}
\usepackage{framed}
\usepackage{url}
\urldef{\mailsa}\path|firstname.lastname@gesis.org|
\newcommand{\keywords}[1]{\par\addvspace\baselineskip
\noindent\keywordname\enspace\ignorespaces#1}

\begin{document}

\mainmatter  % start of an individual contribution

% first the title is needed
\title{Analyzing the research output presented at European NKOS workshops (2000-2015)}

% a short form should be given in case it is too long for the running head


% the name(s) of the author(s) follow(s) next
%
% NB: Chinese authors should write their first names(s) in front of
% their surnames. This ensures that the names appear correctly in
% the running heads and the author index.
%
\author{Fakhri Momeni%
	\and Philipp Mayr}
%
\titlerunning{Analyzing the research output of NKOS workshops}
% (feature abused for this document to repeat the title also on left hand pages)

% the affiliations are given next; don't give your e-mail address
% unless you accept that it will be published
\author{Fakhri Momeni and Philipp Mayr}
\institute{GESIS - Leibniz Institute for the Social Sciences,\\
	Unter Sachsenhausen 6-8\\
	50667 Cologne, Germany\\
	\email{firstname.lastname@gesis.org} }

%
% NB: a more complex sample for affiliations and the mapping to the
% corresponding authors can be found in the file "llncs.dem"
% (search for the string "\mainmatter" where a contribution starts).
% "llncs.dem" accompanies the document class "llncs.cls".
%

%\toctitle{Lecture Notes in Computer Science}
%\tocauthor{Authors' Instructions}
\maketitle


\begin{abstract}		
We analyse the research output which has been presented at the NKOS workshops in the xy papers with xy single author. We present an overview of papers, authors and co-authorships. A focus will be the development of collaboration, core authors and topics in this interdisciplinary field.
 
\keywords{NKOS workshops, Special Issues, Output analysis, Network analysis, Collaboration, Topics}
\end{abstract}


\section{Introduction}\label{intro}
intro %philipp


\section{Data set}\label{dataset}
data set description %philipp

\section{Analysis}\label{analysis}
...


\section{Conclusion}\label{concl}
...

\section{Acknowledgment}\label{sec:ACKNOWLEDGMENTS}
...

\newpage

\bibliographystyle{splncs03} % abbrev
\bibliography{nkos} 

\end{document}
